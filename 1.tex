%!TeX encoding = UTF-8
%!TeX program = xelatex
\documentclass[notheorems, aspectratio=54]{beamer}
\setbeamertemplate{navigation symbols}{}
\useoutertheme{shadow}

\setbeamertemplate{headline}{}
% aspectratio: 1610, 149, 54, 43(default), 32



\usepackage{latexsym}
\usepackage{amsmath,amssymb}
\usepackage{mathtools}
\usepackage{color,xcolor}
\usepackage{graphicx}
\usepackage{algorithm}
\usepackage{amsthm}
\usepackage{lmodern} % 解决 font warning
% \usepackage[UTF8]{ctex}
\usepackage{animate} % insert gif
\usepackage{caption}
\usepackage[backend=biber,autocite=superscript]{biblatex}
\AtBeginBibliography{\small}
\addbibresource{refs.bib}
\captionsetup[figure]{font=scriptsize}

\usepackage{lipsum} % To generate test text 
\usepackage{ulem} % 下划线,波浪线

\usepackage{listings} % display code on slides; don't forget [fragile] option after \begin{frame}

% ----------------------------------------------
% tikx
\usepackage{framed}
\usepackage{tikz}
\usepackage{pgf}
\usetikzlibrary{calc,trees,positioning,arrows,chains,shapes.geometric,%
    decorations.pathreplacing,decorations.pathmorphing,shapes,%
    matrix,shapes.symbols}


\pgfmathsetseed{1} % To have predictable results
% Define a background layer, in which the parchment shape is drawn
\pgfdeclarelayer{background}
\pgfsetlayers{background,main}

% define styles for the normal border and the torn border
\tikzset{
  normal border/.style={black!70!gray, decorate, 
     decoration={random steps, segment length=2.5cm, amplitude=.7mm}},
  torn border/.style={black!70!gray, decorate, 
     decoration={random steps, segment length=.5cm, amplitude=1.7mm}}}

% Macro to draw the shape behind the text, when it fits completly in the
% page
\def\parchmentframe#1{
\tikz{
  \node[inner sep=2em] (A) {#1};  % Draw the text of the node
  \begin{pgfonlayer}{background}  % Draw the shape behind
  \fill[normal border] 
        (A.south east) -- (A.south west) -- 
        (A.north west) -- (A.north east) -- cycle;
  \end{pgfonlayer}}}

% Macro to draw the shape, when the text will continue in next page
\def\parchmentframetop#1{
\tikz{
  \node[inner sep=2em] (A) {#1};    % Draw the text of the node
  \begin{pgfonlayer}{background}    
  \fill[normal border]              % Draw the ``complete shape'' behind
        (A.south east) -- (A.south west) -- 
        (A.north west) -- (A.north east) -- cycle;
  \fill[torn border]                % Add the torn lower border
        ($(A.south east)-(0,.2)$) -- ($(A.south west)-(0,.2)$) -- 
        ($(A.south west)+(0,.2)$) -- ($(A.south east)+(0,.2)$) -- cycle;
  \end{pgfonlayer}}}

% Macro to draw the shape, when the text continues from previous page
\def\parchmentframebottom#1{
\tikz{
  \node[inner sep=2em] (A) {#1};   % Draw the text of the node
  \begin{pgfonlayer}{background}   
  \fill[normal border]             % Draw the ``complete shape'' behind
        (A.south east) -- (A.south west) -- 
        (A.north west) -- (A.north east) -- cycle;
  \fill[torn border]               % Add the torn upper border
        ($(A.north east)-(0,.2)$) -- ($(A.north west)-(0,.2)$) -- 
        ($(A.north west)+(0,.2)$) -- ($(A.north east)+(0,.2)$) -- cycle;
  \end{pgfonlayer}}}

% Macro to draw the shape, when both the text continues from previous page
% and it will continue in next page
\def\parchmentframemiddle#1{
\tikz{
  \node[inner sep=2em] (A) {#1};   % Draw the text of the node
  \begin{pgfonlayer}{background}   
  \fill[normal border]             % Draw the ``complete shape'' behind
        (A.south east) -- (A.south west) -- 
        (A.north west) -- (A.north east) -- cycle;
  \fill[torn border]               % Add the torn lower border
        ($(A.south east)-(0,.2)$) -- ($(A.south west)-(0,.2)$) -- 
        ($(A.south west)+(0,.2)$) -- ($(A.south east)+(0,.2)$) -- cycle;
  \fill[torn border]               % Add the torn upper border
        ($(A.north east)-(0,.2)$) -- ($(A.north west)-(0,.2)$) -- 
        ($(A.north west)+(0,.2)$) -- ($(A.north east)+(0,.2)$) -- cycle;
  \end{pgfonlayer}}}

% Define the environment which puts the frame
% In this case, the environment also accepts an argument with an optional
% title (which defaults to ``Example'', which is typeset in a box overlaid
% on the top border
\newenvironment{parchment}[1][Example]{%
  \def\FrameCommand{\parchmentframe}%
  \def\FirstFrameCommand{\parchmentframetop}%
  \def\LastFrameCommand{\parchmentframebottom}%
  \def\MidFrameCommand{\parchmentframemiddle}%
  \vskip\baselineskip
  \MakeFramed {\FrameRestore}
  \noindent\tikz\node[inner sep=1ex, draw=black!20, fill=black!90, 
          anchor=west, overlay] at (0em, 2em) {\sffamily#1};\par}%
{\endMakeFramed}

% ----------------------------------------------

\mode<presentation>{
    \usetheme{Warsaw}
    % Boadilla CambridgeUS
    % default Antibes Berlin Copenhagen
    % Madrid Montpelier Ilmenau Malmoe
    % Berkeley Singapore Warsaw
    \usecolortheme{seagull}
    % beetle, beaver, orchid, whale, dolphin, seagull
    \useoutertheme{infolines}
    % infolines miniframes shadow sidebar smoothbars smoothtree split tree
    \useinnertheme{circles}
    % circles, rectanges, rounded, inmargin
}

% ---------------------------------------------------------------------
% Jet Black Theme
% \setbeamercolor{normal text}{fg=white,bg=black!90}
% \setbeamercolor{structure}{fg=white}
%
% \setbeamercolor{alerted text}{fg=red!85!black}
%
% \setbeamercolor{item projected}{use=item,fg=black,bg=item.fg!35}
%
% \setbeamercolor*{palette primary}{use=structure,fg=structure.fg}
% \setbeamercolor*{palette secondary}{use=structure,fg=structure.fg!95!black}
% \setbeamercolor*{palette tertiary}{use=structure,fg=structure.fg!90!black}
% \setbeamercolor*{palette quaternary}{use=structure,fg=structure.fg!95!black,bg=black!80}
%
% \setbeamercolor*{framesubtitle}{fg=white}
%
% \setbeamercolor*{block title}{parent=structure,bg=black!70!gray}
% \setbeamercolor*{block body}{fg=black,bg=black!10}
% \setbeamercolor*{block title alerted}{parent=alerted text,bg=black!15}
% \setbeamercolor*{block title example}{parent=example text,bg=black!15}
% ---------------------------------------------------------------------


% ---------------------------------------------------------------------
% flow chart
\tikzset{
    >=stealth',
    punktchain/.style={
        rectangle, 
        rounded corners, 
        % fill=black!10,
        draw=white, very thick,
        text width=6em,
        minimum height=2em, 
        text centered, 
        on chain
    },
    largepunktchain/.style={
        rectangle,
        rounded corners,
        draw=white, very thick,
        text width=10em,
        minimum height=2em,
        on chain
    },
    line/.style={draw, thick, <-},
    element/.style={
        tape,
        top color=white,
        bottom color=blue!50!black!60!,
        minimum width=6em,
        draw=blue!40!black!90, very thick,
        text width=6em, 
        minimum height=2em, 
        text centered, 
        on chain
    },
    every join/.style={->, thick,shorten >=1pt},
    decoration={brace},
    tuborg/.style={decorate},
    tubnode/.style={midway, right=2pt},
    font={\fontsize{10pt}{12}\selectfont},
}
% ---------------------------------------------------------------------

% code setting
\lstset{
    language=C++,
    basicstyle=\ttfamily\footnotesize,
    keywordstyle=\color{red},
    breaklines=true,
    xleftmargin=2em,
    numbers=left,
    numberstyle=\color[RGB]{222,155,81},
    frame=leftline,
    tabsize=4,
    breakatwhitespace=false,
    showspaces=false,               
    showstringspaces=false,
    showtabs=false,
    morekeywords={Str, Num, List},
}

% ---------------------------------------------------------------------

\newcommand{\reditem}[1]{\setbeamercolor{item}{fg=red}\item #1}

% 缩放公式大小
\newcommand*{\Scale}[2][4]{\scalebox{#1}{\ensuremath{#2}}}

% 解决 font warning
\renewcommand\textbullet{\ensuremath{\bullet}}

% -------------------------------------------------------------

%% preamble
\title[Paths to Synchronization on Complex Networks]{Paths to Synchronization on Complex Networks}
% \subtitle{The subtitle}
\author{Jingsheng Gao}
\institute[AQNU]{garrison.null@gmail.com}

% -------------------------------------------------------------

\begin{document}

% title frame
\begin{frame}
    \titlepage
\end{frame}

\begin{frame}{Introduction}
 \begin{itemize}
    \item Synchronization on complex networks is a fascinating phenomenon that explores how interconnected systems can achieve coordinated behavior.
    \item Among collective phenomena in natural and social systems, the synchronization of a set of interacting individuals or units has been intensively studied because of its ubiquity in the natural world.
    \item It is applied on engineering systems, biological systems, social dynamics, and more
  \end{itemize}
\end{frame}

\begin{frame}{Synchronization}
  \begin{figure}
    \centering
  \end{figure}
  \begin{itemize}
    \item Synchronization is the process by which components of a system adjust their behavior to a common rhythm or pattern.
    \item Metronome synchronization, firefly flashing, circadian rhythms, brain networks.
\begin{figure}
  \begin{minipage}[b]{0.45\textwidth}
    \textbf{Schooling Fish}\par\medskip
    \includegraphics[width=0.4\textwidth]{schooling_fish.png}
  \end{minipage}
  \hfill
  \begin{minipage}[b]{0.45\textwidth}
    \textbf{Bioluminescence in Oceans}\par\medskip
    \includegraphics[width=0.4\textwidth]{bioluminescence.png}
  \end{minipage}
\end{figure}
  \end{itemize}
\end{frame}


\begin{frame}{Complex Networks}
  \begin{itemize}
    \item A complex network is a graph (network) with non-trivial topological features—features that do not occur in simple networks such as lattices or random graphs but often occur in networks representing real systems.
    \item  The study of complex networks is a young and active area of scientific research (since 2000) inspired largely by empirical findings of real-world networks such as computer networks, biological networks, technological networks, brain networks, climate networks and social networks.
  \end{itemize}
\end{frame}

\begin{frame}{Factors Influencing Synchronization}
  \begin{itemize}
    \item Network Topology: The structure of connections, such as of small-world network, scale-free network, Erdős–Rényi model, significantly impacts synchronization.
    \item Coupling Strength: The intensity of interactions between nodes plays a crucial role. Stronger coupling generally promotes synchronization.
    \item Node Dynamics: The intrinsic properties of individual nodes, such as their natural frequencies or response patterns, also influence synchronization.
  \end{itemize}
\end{frame}

\begin{frame}{Scale-free network}
  \begin{itemize}
   \item A scale-free network\cite{wiki:scale_free_network} is a network whose degree distribution follows a power law, at least asymptotically.
  \end{itemize}
  \begin{figure}
    \centering
    \textbf{Examples}\par\medskip
    \includegraphics[width=0.4\textwidth]{scale_free_network.png}
  \end{figure}
  \begin{figure}
    \centering
    \textbf{Degree distribution}\par\medskip
    \includegraphics[width=0.4\textwidth]{scale_free_distribution.png}
    \caption{Degree distribution for a network with 150000 vertices and mean degree = 6 created using the Barabási–Albert model (blue dots). The distribution follows an analytical form given by the ratio of two gamma functions (black line) which approximates as a power-law.}
  \end{figure}
\end{frame}

\begin{frame}{Erdős–Rényi model}
  \begin{itemize}
    \item In the mathematical field of graph theory, the Erdős–Rényi model refers to one of two closely related models for generating random graphs or the evolution of a random network.
  \end{itemize}
  \begin{figure}
    \centering
    \textbf{Partial map of the Internet}\par\medskip
    \includegraphics[width=0.4\textwidth]{internet_map.jpg}
      \caption{Each line is drawn between two nodes, representing two IP addresses. The length of the lines are indicative of the delay between those two nodes. This graph represents less than 30\% of the Class C networks reachable by the data collection program in early 2005.}
  \end{figure}
\end{frame}

\begin{frame}{Generating methods of Erdős–Rényi model}
  \begin{itemize}
    \item In the $G(n, M)$ model, a graph is chosen uniformly at random from the collection of all graphs which have $n$ nodes and M edges. For example, in the $G(3,2)$ model, there are three two-edge graphs on three labeled vertices (one for each choice of the middle vertex in a two-edge path), and each of these three graphs is included with probability.
    \item In the $G(n, p)$ model, a graph is constructed by connecting labeled nodes randomly. Each edge is included in the graph with probability $p$, independently from every other edge. Equivalently, the probability for generating each graph that has $n$ nodes and $M$ edges is ${\displaystyle p^{M}(1-p)^{{n \choose 2}-M}}$.
  \begin{figure}
    \centering
    \includegraphics[width=0.4\textwidth]{er_example.jpg}
      \caption{A graph generated by the binomial model of Erdős and Rényi (p = 0.01)}
  \end{figure}
  \end{itemize}
\end{frame}

\begin{frame}{References}
    \printbibliography
\end{frame}

\end{document}


